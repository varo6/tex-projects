\documentclass{article}

% Language setting
% Replace `english' with e.g. `spanish' to change the document language
\usepackage[spanish]{babel}

% Set page size and margins
% Replace `letterpaper' with `a4paper' for UK/EU standard size
\usepackage[a4paper,top=2cm,bottom=2cm,left=3cm,right=3cm,marginparwidth=1.75cm]{geometry}

% Useful packages
\usepackage{amsmath}
\usepackage{graphicx}
\usepackage[colorlinks=true, allcolors=blue]{hyperref}
\usepackage{fancybox}

\title{Apuntes y ejercicios random de matemáticas}
\author{}
\date{04 de Febrero de 2024}

\begin{document}
\maketitle


\begin{abstract}

Ejercicios o fórmulas que he pasado a LaTeX para dar clases particulares. Ayudas/Fórmulas, no está ordenado ni tiene por qué estar correctamente. Como un borrador.

\end{abstract}


\section{Ejercicios}


\boxed{1} Dados dos planos:
\begin{gather*}
    \pi_1 :  4x + 6y - 12z + 1 = 0 \quad  \quad \pi_2 : -2x - 3y + 6z - 5 = 0
\end{gather*}

\begin{enumerate}
    \setcounter{enumi}{0} % Establece el contador de la lista en 0
    \item[a)] Halla el volumen que formaría un paralelepípedo que tiene como dos de sus caras $\pi_1$ y $\pi_2$.
    \item[b)] ¿Y si en vez de un paralelepípedo fuera un cubo?
\end{enumerate}

\quad
\quad
\quad

\boxed{2} Considere la recta y los planos:
\begin{gather*}
    r :  \frac{x-2}{-1} = \frac{y-2}{3} = \frac{z-1}{1}\quad  \quad \pi_1 : x = 0 \quad  \quad \pi_2 : y = 0
\end{gather*}
\begin{enumerate}
    \setcounter{enumi}{0} % Establece el contador de la lista en 0
    \item[a)] Halla los puntos de la recta r que equidistan de los planos $\pi_1$ y $\pi_2$.
    \item[b)] Determina la posición relativa de la recta r y la recta intersección de los planos $\pi_1$ y $\pi_2$.
\end{enumerate}


\section{4 ESO ayuda}

\subsection{Algunas identidades trigonométricas}

\begin{align*}
    \sec x &= \frac{1}{\cos x} & \csc x &= \frac{1}{\sin x} \\
    \quad \\
    \cot x &= \frac{\cos x}{\sin x} & \tan x &= \frac{\sin x}{\cos x} \\
    \quad \\
    \sin^2 x + \cos^2 x &= 1 & \tan^2 x + 1 &= \frac{1}{\cos^2  x}
\end{align*}

\subsection{Fórmulas del triángulo}

\begin{align*}
    \sin x &= \frac{\text{Cateto Opuesto}}{\text{Hipotenusa}} & \csc{x } &= \frac{\text{Hipotenusa}}{\text{Cateto Opuesto}} \\
    \quad \\
    \cos x &= \frac{\text{Cateto Contiguo}}{\text{Hipotenusa}} & \sec x &= \frac{\text{Hipotenusa}}{\text{Cateto Contiguo}} \\
    \quad \\
    \tan x &= \frac{\text{Cateto Opuesto}}{\text{Cateto Contiguo}} & \cot x &= \frac{\text{Cateto Contiguo}}{\text{Cateto Opuesto}} \\
\end{align*}


\noindent $$\text{RECORDATORIO DE PITÁGORAS: Hipotenusa}^2 = \text{Contiguo} ^2 + \text{Opuesto}^2$$



\subsection{Razones Trigonométricas}
\begin{table}[h]
    \centering
    \large
    \renewcommand{\arraystretch}{2} % Ajusta el espaciado vertical
    \begin{tabular}{|c|c|c|c|}
        \hline
        & $30^\circ$ & $45^\circ$ & $60^\circ$ \\
        \hline
        $\sin$ & $\frac{1}{2}$ & $\frac{\sqrt{2}}{2}$ & $\frac{\sqrt{3}}{2}$ \\
        \hline
        $\cos$ & $\frac{\sqrt{3}}{2}$ & $\frac{\sqrt{2}}{2}$ & $\frac{1}{2}$ \\
        \hline
        $\tan$ & $\frac{1}{\sqrt{3}}$ & $1$ & $\sqrt{3}$ \\
        \hline
    \end{tabular}
    \caption{Razones trigonométricas para ángulos comunes}
    \label{tab:trig_ratios}
\end{table}
\end{document}