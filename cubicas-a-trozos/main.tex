\documentclass{article}

% Idioma
\usepackage[spanish]{babel}

% márgenes y formato
\usepackage[a4paper,top=2cm,bottom=2cm,left=3cm,right=3cm,marginparwidth=1.75cm]{geometry}
\usepackage{parskip}

% Otros paquetes importantes
\usepackage{amsmath}
\usepackage{amssymb}
\usepackage{graphicx}
\usepackage[colorlinks=true, linkcolor=black, urlcolor=blue]{hyperref}

% PARA APUNTES: darkmode
% \usepackage{darkmode}
% \enabledarkmode

% Titulo y autor
\title{Cúbicas a Trozos}
\author{Álvaro Hernández Riquelme}
\date{\today}

\begin{document}

%%%%%%%%%%%%%%%%%%%%%%%%%%%%%%%%%%

\maketitle
\tableofcontents
\newpage

\section{Introducción}

Nuestra función será
\begin{equation}
f(x) = 3x^4 - x^3 - 3x + 1
\end{equation}

con los nodos $x_0 = 0$, $x_1 = 2$.

\subsection{Obtener la cota de error}

Se nos pide obtener la cota de error $|f(x)-H(x)|$ válida para $0<=2<=1$.

Lo primero que haremos sería obtener la cuarta derivada, que, listando cada una, será:

\begin{alignat*}{2}
&f'(x) &&= 12x^3 - 3x^2 - 3 \\
&f''(x) &&= 36x^2 - 6x \\
&f'''(x) &&= 72x - 6 \\
&f^{(4)}(x) &&= 72
\end{alignat*}

Para finalmente, obtener la cota de error con la formula del pdf, siendo la ecuación $16$:

\begin{equation}
|f(x) - H(x)| \leq C_4 \cdot \frac{h^4}{384}
\end{equation}

Donde $C_4$ es el máximo de la cuarta derivada en el intervalo $[0,2]$, que en este caso es $72$, y h es la distancia entre los nodos, que en este caso es $2$.

Por lo tanto, sustituyendo en la ecuación, tenemos:

\begin{equation}
|f(x) - H(x)| \leq 72 \cdot \frac{2^4}{384}
\end{equation}

\begin{equation}
|f(x) - H(x)| \leq \frac{1152}{384} = 3
\end{equation}

La cota de error es de \boxed{$3$}.

\subsection{Calcular H(x) mediante la Forma de Newton}

En este apartado se nos pide calcular el polinomio de Hermite $H(x)$, por lo que, usaremos las fórmulas vistas en clase, y que se encuentran en el PDF de teoría. Para ello, iremos paso por paso hasta tener todos los valores y poder sustituir en la fórmula de Newton (en el último paso).

Calculamos la primera derivada, necesaria para las diferencias divididas:
$$ f'(x) = \frac{d}{dx}(3x^4 - x^3 - 3x + 1) = 12x^3 - 3x^2 - 3 $$

Evaluamos \(f(x)\) y \(f'(x)\) en los nodos \(a=0\) y \(b=2\):
\begin{itemize}
    \item \(f(0) = 3(0)^4 - (0)^3 - 3(0) + 1 = 1\)
    \item \(f(2) = 3(2)^4 - (2)^3 - 3(2) + 1 = 3(16) - 8 - 6 + 1 = 48 - 8 - 6 + 1 = 35\)
    \item \(f'(0) = 12(0)^3 - 3(0)^2 - 3 = -3\)
    \item \(f'(2) = 12(2)^3 - 3(2)^2 - 3 = 12(8) - 3(4) - 3 = 96 - 12 - 3 = 81\)
\end{itemize}

Los coeficientes del polinomio de Hermite en la forma de Newton son las diferencias divididas \(f[0]\), \(f[0,0]\), \(f[0,0,2]\), y \(f[0,0,2,2]\). Las calculamos según la fórmula ($3$) del PDF de teoría.

\begin{itemize}
    \item \(f[0] = f(0) = 1\)
    \item \(f[0,0] = f'(0) = -3\)
    \item \(f[2] = f(2) = 35\)
    \item \(f[2,2] = f'(2) = 81\)
\end{itemize}

\begin{itemize}
    \item \(f[0,2] = \dfrac{f(2) - f(0)}{2-0} = \dfrac{35 - 1}{2} = \dfrac{34}{2} = 17\)
\end{itemize}

Seguimos con la fórmula mencionada ($3$) del PDF, para las siguientes diferencias divididas, usando valores calculados en las anteriores diferencias:

\begin{itemize}
    \item \(f[0,0,2] = \dfrac{f[0,2] - f[0,0]}{2-0} = \dfrac{17 - (-3)}{2} = \dfrac{20}{2} = 10\)
    \item  \(f[0,2,2] = \dfrac{f[2,2] - f[0,2]}{2-0} = \dfrac{81 - 17}{2} = \dfrac{64}{2} = 32\)
\end{itemize}

Finalmente, con los valores obtenidos, calculamos la última diferencia dividida:

\begin{itemize}
    \item \(f[0,0,2,2] = \dfrac{f[0,2,2] - f[0,0,2]}{2-0} = \dfrac{32 - 10}{2} = \dfrac{22}{2} = 11\)
\end{itemize}

Usando la fórmula de Newton con los nodos \(0,0,2,2\) y los coeficientes calculados:

\begin{equation}
H(x) = f[a] + f[a,a](x-a) + f[a,a,b](x-a)^2 + f[a,a,b,b](x-a)^2(x-b)
\end{equation}
\begin{equation}
H(x) = f[0] + f[0,0](x-0) + f[0,0,2](x-0)(x-0) + f[0,0,2,2](x-0)(x-0)(x-2)
\end{equation}
Sustituyendo los valores de los coeficientes:
\begin{equation}
H(x) = 1 + (-3)x + 10x^2 + 11x^2(x-2)
\end{equation}

Si lo expandimos, la solución finalmente es:

$$ \boxed{H(x) = 11x^3 - 12x^2 - 3x + 1} $$

\subsection{Calcula S(x) resolviendo el sistema de ecuaciones}

Para un único intervalo \([x_0, x_1]\), el polinomio del spline, \(S_0(x)\), se define según la fórmula ($24$) del PDF de la teoría:
\begin{equation}
S_0(x) = \alpha_{0,0} + \alpha_{0,1}(x-x_0) + \alpha_{0,2}(x-x_0)^2 + \alpha_{0,3}(x-x_0)^3
\end{equation}

Necesitamos determinar los cuatro coeficientes \(\alpha_{0,0}, \alpha_{0,1}, \alpha_{0,2}, \alpha_{0,3}\). Para ello, hago sistema de cuatro ecuaciones lineales basado en las siguientes condiciones:

\begin{enumerate}
    \item El spline debe pasar por el punto \((x_0, f(x_0))\). Esto lo saco de la ecuación ($26$) del PDF, que sería:
    $$ \alpha_{0,0} = f(x_0) $$

    \item El spline debe pasar por el punto \((x_1, f(x_1))\). Esto se deriva de la segunda parte de la ecuación ($26$) :
    $$ \alpha_{0,0} + \alpha_{0,1}(x_1-x_0) + \alpha_{0,2}(x_1-x_0)^2 + \alpha_{0,3}(x_1-x_0)^3 = f(x_1) $$

    \item Para un spline natural, la segunda derivada en los extremos del intervalo global es cero. Esto seencuentra en la ecuación ($28$), \(S''(x_0) = 0\). La segunda derivada de \(S_0(x)\) es \(S_0''(x) = 2\alpha_{0,2} + 6\alpha_{0,3}(x-x_0)\). Al evaluar en \(x=x_0\), obtenemos:
    $$ S_0''(x_0) = 2\alpha_{0,2} = 0 \implies \alpha_{0,2} = 0 $$
    Esta es la primera de las ``ecuaciones extra'' mencionadas en la ecuación ($27$).

    \item la ecuación ($28$) exige que \(S''(x_1) = 0\). Al evaluar \(S_0''(x)\) en \(x=x_1\), obtenemos:
    $$ S_0''(x_1) = 2\alpha_{0,2} + 6\alpha_{0,3}(x_1-x_0) = 0 $$
    Esta es la segunda de las ``ecuaciones extra''.
\end{enumerate}


Ahora pasamos a plantear el sistema de ecuaciones.
Primero, evaluamos la función \(f(x) = 3x^4 - x^3 - 3x + 1\) en los nodos \(x_0=0\) y \(x_1=2\):
\begin{itemize}
    \item \(f(0) = 3(0)^4 - (0)^3 - 3(0) + 1 = 1\)
    \item \(f(2) = 3(2)^4 - (2)^3 - 3(2) + 1 = 3(16) - 8 - 6 + 1 = 35\)
\end{itemize}

Ahora, construimos el sistema de 4 ecuaciones con 4 incógnitas (\(\alpha_{0,0}, \alpha_{0,1}, \alpha_{0,2}, \alpha_{0,3}\)) usando las fórmulas deducidas:
\begin{enumerate}
    \item \(\alpha_{0,0} = f(0) = 1\)
    \item \(\alpha_{0,0} + \alpha_{0,1}(2-0) + \alpha_{0,2}(2-0)^2 + \alpha_{0,3}(2-0)^3 = f(2)\) \\
    \(\implies \alpha_{0,0} + 2\alpha_{0,1} + 4\alpha_{0,2} + 8\alpha_{0,3} = 35\)
    \item \(\alpha_{0,2} = 0 \)
    \item \(2\alpha_{0,2} + 6\alpha_{0,3}(2-0) = 0 \implies 2\alpha_{0,2} + 12\alpha_{0,3} = 0\)
\end{enumerate}

Ahora resolveremos por sustitución ya que ya tenemos \(\alpha_{0,2}\) y \(\alpha_{0,0}\):
\begin{itemize}
    \item Sustituimos \(\alpha_{0,2}=0\) :
    \begin{align*}
    &2(0) + 12\alpha_{0,3} = 0 &&\\
    &\alpha_{0,3} = 0 &&
    \end{align*}
    \item Finalmente, sustituimos los valores de \(\alpha_{0,0}\), \(\alpha_{0,2}\) y \(\alpha_{0,3}\):
    \begin{align*}
    &(1) + 2\alpha_{0,1} + 4(0) + 8(0) = 35 &&\\
    &1 + 2\alpha_{0,1} = 35 &&\\
    &2\alpha_{0,1} = 34 &&\\
    &\alpha_{0,1} = 17 &&
    \end{align*}
\end{itemize}
Los coeficientes son: \(\alpha_{0,0} = 1\), \(\alpha_{0,1} = 17\), \(\alpha_{0,2} = 0\), \(\alpha_{0,3} = 0\).

Sustituimos los coeficientes en la fórmula:

$$ S(x) = S_0(x) = \alpha_{0,0} + \alpha_{0,1}(x-x_0) + \alpha_{0,2}(x-x_0)^2 + \alpha_{0,3}(x-x_0)^3 $$
$$ S(x) = 1 + 17(x-0) + 0(x-0)^2 + 0(x-0)^3 $$
$$ \boxed{S(x) = 17x + 1} $$

\subsection{Recalcular el nuevo $H(x)$}

Ahora se nos pide recalcular el polinomio de Hermite $H(x)$ si añadimos un nuevo nodo \(x_2 = 3\), aunque reutilizaremos calculos hechos anteriormente.

Una de las propiedades más poderosas de la forma de Newton es la \textbf{permanencia}. Los coeficientes calculados para los nodos \(0,0,2,2\) no cambian. Simplemente calcularemos los nuevos coeficientes que surgen al añadir los nodos \(3,3\).

\textbf{Recapitulando los coeficientes anteriores:}
\begin{itemize}
    \item \(f[0] = 1\)
    \item \(f[0,0] = -3\)
    \item \(f[0,0,2] = 10\)
    \item \(f[0,0,2,2] = 11\)
\end{itemize}

El polinomio anterior era \(H_{[0,2]}(x) = 1 - 3x + 10x^2 + 11x^2(x-2)\). Nuestro nuevo polinomio será una extensión de este. Los pasos serán similares a los del ejercicio 1.

Primero, evaluamos la función y su derivada en  \(x=3\).
\begin{itemize}
    \item \(f(3) = 3(3)^4 - (3)^3 - 3(3) + 1 = 3(81) - 27 - 9 + 1 = 243 - 27 - 9 + 1 = 208\)
    \item \(f'(3) = 12(3)^3 - 3(3)^2 - 3 = 12(27) - 3(9) - 3 = 324 - 27 - 3 = 294\)
\end{itemize}

\subsection*{3. Extensión de la Tabla de Diferencias Divididas}
Ahora, extendemos la tabla para incluir los nuevos valores \(f[3]=208\) y \(f[3,3]=f'(3)=294\). Calculamos las nuevas diferencias divididas necesarias.

\begin{itemize}
    \item \(f[2,3] = \dfrac{f[3] - f[2]}{3-2} = \dfrac{208 - 35}{1} = 173\)
\end{itemize}

\begin{itemize}
    \item \(f[2,2,3] = \dfrac{f[2,3] - f[2,2]}{3-2} = \dfrac{173 - 81}{1} = 92\)
    \item \(f[2,3,3] = \dfrac{f[3,3] - f[2,3]}{3-2} = \dfrac{294 - 173}{1} = 121\)
\end{itemize}

\begin{itemize}
    \item \(f[0,2,2,3] = \dfrac{f[2,2,3] - f[0,2,2]}{3-0} = \dfrac{92 - 32}{3} = \dfrac{60}{3} = 20\)
    \item \(f[2,2,3,3] = \dfrac{f[2,3,3] - f[2,2,3]}{3-2} = \dfrac{121 - 92}{1} = 29\)
\end{itemize}

\begin{itemize}
    \item \(f[0,0,2,2,3] = \dfrac{f[0,2,2,3] - f[0,0,2,2]}{3-0} = \dfrac{20 - 11}{3} = \dfrac{9}{3} = 3\)
\end{itemize}

Por lo que uno de los nuevos coeficientes que usaremos es $f[0,0,2,2,3] = 3$.

Para calcular el último coeficiente, necesitamos una diferencia más de 4º orden:
\begin{itemize}
    \item \(f[0,2,2,3,3] = \dfrac{f[2,2,3,3] - f[0,2,2,3]}{3-0} = \dfrac{29 - 20}{3} = \dfrac{9}{3} = 3\)
\end{itemize}
Y finalmente:
\begin{itemize}
    \item \(f[0,0,2,2,3,3] = \dfrac{f[0,2,2,3,3] - f[0,0,2,2,3]}{3-0} = \dfrac{3 - 3}{3} = 0\)
\end{itemize}
Este es nuestro último coeficiente, que será \(f[0,0,2,2,3,3] = 0\).


\quad


Sustituimos aprovechando los coeficientes anteriores:

$$ H(x) = \underbrace{f[0] + f[0,0]x + f[0,0,2]x^2 + f[0,0,2,2]x^2(x-2)}_{\text{Polinomio anterior}} + f[0,0,2,2,3]x^2(x-2)^2 + f[0,0,2,2,3,3]x^2(x-2)^2(x-3) $$
$$ H(x) = (11x^3 - 12x^2 - 3x + 1) + (3)x^2(x-2)^2 + (0)x^2(x-2)^2(x-3) $$

Ahora, simplificamos la expresión:

\begin{align*} H(x) &= 11x^3 - 12x^2 - 3x + 1 + 3x^2(x^2 - 4x + 4) \\ &= 11x^3 - 12x^2 - 3x + 1 + 3x^4 - 12x^3 + 12x^2 \\ &= 3x^4 - x^3 - 3x + 1 \end{align*}

$$ \boxed{H(x) = 3x^4 - x^3 - 3x + 1} $$


\subsection{Verificación de la Continuidad de la Función y su Derivada}
Ahora comprobamos si se cumplen las igualdades que se muestran en \eqref{eq:continuity}, que la primera coincidiría con la comprobación de la continuidad en $x=2$ de la función y la segunda la de la derivada.

\begin{equation}
\label{eq:continuity}
\begin{aligned}
H(2^-) &= H(2^+) = f(2) \\
H'(2^-) &= H'(2^+) = f'(2)
\end{aligned}
\end{equation}

La función \(f(x)\) y su derivada \(f'(x)\) las tenemos, y sus valores en el nodo \(x=2\) son:

\begin{itemize}
    \item \(f(2) = 3(16) - 8 - 6 + 1 = 35\)
    \item \(f'(2) = 12(8) - 3(4) - 3 = 96 - 12 - 3 = 81\)
\end{itemize}

Ahora calculamos el polinomio a la izquierda, del primer ejercicio tenemos:

$$ H_0(x) = 11x^3 - 12x^2 - 3x + 1 $$
$$ H'_0(x) = 33x^2 - 24x - 3 $$

Calculamos los coeficientes para la derecha (algunos ya los teníamos del ejercicio anterior):
\begin{itemize}
    \item \(f[2] = f(2) = 35\)
    \item \(f[2,2] = f'(2) = 81\)
    \item \(f[2,2,3] = \frac{f[2,3] - f[2,2]}{3-2} = \frac{173 - 81}{1} = 92\)
    \item \(f[2,2,3,3] = \frac{f[2,3,3] - f[2,2,3]}{3-2} = \frac{121 - 92}{1} = 29\)
\end{itemize}
Por lo tanto, el polinomio es:
$$ H_1(x) = 35 + 81(x-2) + 92(x-2)^2 + 29(x-2)^2(x-3) $$
Y su derivada es:
$$ H'_1(x) = 81 + 184(x-2) + 58(x-2)(x-3) + 29(x-2)^2 $$

Ahora comprobamos las igualdades de continuidad y derivada en \(x=2\):

$$ H(2^-) = H_0(2) = 11(2)^3 - 12(2)^2 - 3(2) + 1  = 88 - 48 - 6 + 1 = \mathbf{35} $$
$$ H(2^+) = H_1(2) = 35 + 81(2-2) + 92(2-2)^2 + 29(2-2)^2(2-3) =  \mathbf{35} $$
$$ f(2) = \mathbf{35} $$
Conclusión: Se cumple que \(H(2^-) = H(2^+) = f(2) = 35\). Esto es así por construcción, ya que ambos polinomios, \(H_0(x)\) y \(H_1(x)\), están forzados a pasar por el punto \((2, f(2))\).

Ahora hacemos las derivadas:
$$ H'(2^-) = H'_0(2) = 33(2)^2 - 24(2) - 3 = 33(4) - 48 - 3 = 132 - 48 - 3 = \mathbf{81} $$
$$ H'(2^+) = H'_1(2) = 81 + 184(2-2) + 58(2-2)(2-3) + 29(2-2)^2 = 81 + 0 + 0 + 0 = \mathbf{81} $$
$$ f'(2) = \mathbf{81} $$
Conclusión: Se cumple que \(H'(2^-) = H'(2^+) = f'(2) = 81\). Esto también es así por construcción. El método de Hermite impone explícitamente que la pendiente de cada polinomio en el nodo de unión sea igual a la pendiente de la función original en ese punto, garantizando una transición suave.

% \vspace{1cm}
% \begin{center}
% \fbox{\parbox{0.8\textwidth}{
% Ambas igualdades se han verificado. Esto demuestra que la interpolación cúbica de Hermite a trozos genera una función que no solo es continua, sino que también tiene una derivada continua en los nodos interiores, lo que resulta en una curva sin ``picos'' o ``esquinas''.
% }}
% \end{center}

\end{document}
